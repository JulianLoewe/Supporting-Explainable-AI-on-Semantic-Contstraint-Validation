The chapter tackled two problems: Validating constraints over machine learning models and using constraint validation results to interpret and, when possible, explain machine learning model predictions. 

Therefore, a model-agnostic validation engine was proposed, which can validate prediction and data constraints. The evaluation of both types of constraints uses the SHACL constraint validation as a central component, which allows using the semantic context of the entities in the knowledge graph when defining the constraints. The sample-to-node mapping allows for an association of the seed nodes (i.e., the entities of interest) and their SHACL constraint validation results with the samples in the dataset. The mapping is retrieved simultaneously with the dataset from a knowledge graph using a dataset generating query. The dataset generating query can be used to create the seed query, which only retrieves the seed nodes. Further, both types of constraints are defined given a target shape, which is evaluated over the seed nodes associated with the samples in the dataset.

The data constraint does not require any further information for evaluation, but the prediction constraints additionally consider the model's predictions. This is also why data constraints can only help interpret, while prediction constraints can also be used to explain model predictions in some cases. The evaluation is done using a 3-valued logic, and unlike the 2-valued one, it allows the identification of samples that are only valid because the condition of the constraint does not apply.

Several heuristics are proposed to speed up the execution of the validation engine: Three heuristics are proposed to accelerate the SHACL constraint validation by minimizing the number of SHACL constraint validation results that need to be produced. The assignment of SHACL constraint validation results to samples in the dataset is analyzed in terms of the join strategy with physical join operators, resulting in two heuristics to speed up the assignment. Finally, the theory behind a SPARQL query engine is presented, which pushes down the SHACL constraint validation and the joining of the SHACL constraint validation results with the samples in the dataset into the SPARQL query execution.

Although the heuristics may help optimize the validation engine's execution time, time-complexity-wise, the theoretical approach is PSPACE-complete as querying the dataset is PSPACE-complete, and the SHACL constraint validation is NP-complete in general. However, reasonable decisions regarding the dataset generating query and the constraints make the problem solvable in polynomial time. 

The results of the validation engine are stored in a model-validation-result function. Frequency distribution tables are consulted for the summarization of a single constraint. The constraint validation results of multiple constraints are further cumulated by a concept called coverage. Coverage reduces the constraint validation results to the number of samples in the dataset through prioritizing constraints and validation results (i.e., invalid over valid over not applicable results). As an application, confusion matrices and decision trees are annotated with the constraint validation results, and general interpretations based on prediction constraints are proposed. The decision tree visualization is inspired by the dtreeviz visualization, but extended to visualize the constraint validation results. Classification, as well as regression tasks, are supported. As the decision tree is composed of constant models, constraints can also be evaluated per node. Both data and prediction constraints can be used to detect patterns learned by the model. 