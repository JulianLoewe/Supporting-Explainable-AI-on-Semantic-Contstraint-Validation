
    As shown graphically in Figure \ref{fig:overall_approach}, the approach described in this chapter can be represented sequentially. The first three steps are the ones that build the foundation and contribute to an approach to solve the problem described in lemma \ref{S:problem_validating_constraints_over_ai_model}. This is also the problem tackled in this section. The initial situation assumed is that a user wants to solve a task (see definition \ref{Def:ml_task}) using a machine learning model based on data available in a knowledge graph (see definition \ref{Def:knowledge_graph}), but the machine learning model should respect a given set of user-created constraints as defined in definition \ref{Def:constraint}. In the first subsection, the first step called \glqq Propositionalization\grqq{} is described in more detail and will provide a dataset (see definition \ref{Def:dataset}) and a sample-to-node mapping (see definition \ref{Def:sample_to_node_mapping}). Building on that, a machine learning model is trained on the dataset, which completes the inputs of the problem tackled and leads to the third step called \glqq Constraint Validation\grqq{} (see Figure \ref{fig:overall_approach}). This step is described in detail in section \ref{section_evaluating_constraints} by presenting the semantics of the constraints. This leads to the validation engine solving the problem model-agnostic.
