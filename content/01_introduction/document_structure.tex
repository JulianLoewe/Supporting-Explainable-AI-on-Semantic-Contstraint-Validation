The thesis is split up into eight chapters. Chapter \ref{section_introduction} is the introduction and provides the reader with a motivating example, the main contributions, and the problem tackled in this work. Next, chapter \ref{section_background} presents the mathematical notations, symbols, and the basic concepts needed to understand the following chapters. Having understood the necessary background, chapter \ref{section_related_work} highlights topics related to this work, summarizes the state-of-the-art while referring to suitable papers, and positions the thesis with respect to them. The approach, including the formal problem definition, the proposed solution, and extensions and improvements to it, are discussed formally in chapter \ref{section_approach}. For a better understanding of the approach, the motivating example is used to demonstrate the method by using it as a running example. Chapter \ref{section_implemenation} provides the reader with an implementation of the approach, highlighting the different modules and design decisions. Parallel to the creation of this work, a resource called ``InterpretME'' has been created as a collaborative project with four Scientific Data Management Research Group members. InterpretME \cite{interpretME} is shortly presented as an application of this work in chapter \ref{section_interpretme}. The implementation is evaluated with respect to its performance based on four benchmarks in chapter \ref{section_evaluation}. Finally, it comes to a conclusion in chapter \ref{section_conclusions}. 