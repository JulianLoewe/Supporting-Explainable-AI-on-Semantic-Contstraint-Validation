Chapter \ref{section_approach}  explained the approach by showing its logical components. These components were put together in pseudocode to allow for a theoretical understanding. The steps were supplemented with the practical application in case of the motivating example. This chapter enriches the theoretical components with a practical implementation written in Python \cite{python}. The implementation is publicly available on GitHub\footnote{\href{https://github.com/JulianLoewe/Validating_Models}{https://github.com/JulianLoewe/Validating\_Models}}.

Python is chosen for its simple syntax, readability, and extensive support for machine learning, as the most popular machine learning frameworks, are written in Python \cite{state_of_data_science_2021}. However, Python comes with the overhead of interpreting and manipulating Python objects, which is why performance-optimized libraries for Python as NumPy \cite{numpy}, pandas \cite{pandas}, and scikit-learn \cite{sklearn_api,scikit-learn} are based on pre-compiled functions written in the C language. 
Similar to the examples, the implementation will also be centered on decision trees, specifically the decision trees generated by the scikit-learn library.