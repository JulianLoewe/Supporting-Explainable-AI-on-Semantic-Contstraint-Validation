The approach proposed in chapter \ref{section_approach} is realized in a library implemented using Python. The implementation is split into two parts the constraint validation engine and the visualization of the validation results in a model-coherent way.

The validation engine makes heavy use of pandas data frames and NumPy vectorization while implementing the heuristics proposed in section \ref{section_shaclapi} (i.e., using the shaclAPI) and in section \ref{section_executing_the_join} (i.e., the implementation of the join strategies using the pandas join function). The \textsc{Dataset} module is split into the \emph{BaseDataset} and the \emph{ProcessDataset} to allow for dataset preparation (e.g., data preprocessing, sampling, feature engineering) while keeping the sample-to-node mapping intact. 

The visualization part is implemented based on the logic provided by the dtreeviz library. Nevertheless, the logic is extended to support multiprocessing and annotation with multiple constraints. Frequency distribution tables are introduced to split the summarizing stage from the visualization stage and allow for the implemented parallelization of the plot generation. Finally, parts of the algorithm were improved through vectorization with NumPy (e.g., the decision-tree-node-to-sample mapping generation) and caching of intermediate results (i.e., SHACL validation results, constraint validation results, coverage results, and model predictions).

The whole implementation is built for extensibility with respect to new constraint types, the usage of other SHACL engines, the group functions used during summarization, visualizations for new model types, and the library used to train the machine learning model.