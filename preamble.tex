\usepackage[utf8]{inputenc}
\usepackage[english]{babel}
\usepackage[T1]{fontenc}
\usepackage[numbers,square]{natbib}
\usepackage[a4paper, margin=1.2in]{geometry}
\usepackage{pdfpages}
%\usepackage{showframe}
\usepackage{booktabs}
\usepackage{xcolor}
\usepackage{graphicx}
\usepackage{tikz}
\usetikzlibrary{shapes,snakes,patterns, positioning}
\usepackage{caption}
\usepackage{subcaption}
\usepackage{afterpage}
\usepackage{tablefootnote}
%\KOMAoption{DIV}{12} %instead of ,left=2cm,right=2cm,top=2cm,bottom=2.2cm etc
%\addtokomafont{disposition}{\rmfamily} % serifes in headings

\pagestyle{plain}
%\addtokomafont{section}{\clearpage}

%\addtolength{\footskip}{-1.5cm}

%%Mathe - Imports
\usepackage{amsmath}
\usepackage{amsfonts}
\usepackage{amssymb}
\usepackage{mathrsfs}
\usepackage{mathtools}
\usepackage{amsthm}
\usepackage[theorems,many]{tcolorbox}
\tcbuselibrary{breakable}
\allowdisplaybreaks
\usepackage[lightning]{stmaryrd}
\usepackage{extarrows}

\usepackage{lmodern}
\usepackage[hidelinks]{hyperref}
\usepackage{float}
\usepackage{enumitem}

\setlength{\parindent}{0pt}

\newcommand{\uri}[1]{\texttt{#1}}

\usepackage[nodayofweek]{datetime}

%% Mathe - Formatierungszeugs
\theoremstyle{definition}
\newtcbtheorem[]{BspI}{Example}{%
	before skip = 11pt,
	colback=blue!5,
	colframe=blue!35!black,
	fonttitle=\bfseries,
	breakable,
	enhanced jigsaw,
	lines before break=3
	}{Bsp}

\definecolor{block-gray}{gray}{0.95}

\newtcbtheorem[]{DefI}{Definition}{%
    colback=white,
    colframe=white,
    %grow to right by=-10mm,
    %grow to left by=-10mm, 
    boxrule=0pt,
    boxsep=0pt,
    breakable,
    enhanced jigsaw,
    borderline west={3pt}{0pt}{black},
    %colbacktitle={block-gray},
    coltitle={black},
    fonttitle={\bfseries},
    attach title to upper={\\},
    lines before break=3
    }{Def}

\newtcbtheorem[]{SatzI}{Lemma}{%
    colback=white,
    colframe=white,
    %grow to right by=-10mm,
    %grow to left by=-10mm, 
    boxrule=0pt,
    boxsep=0pt,
    breakable,
    enhanced jigsaw,
    borderline west={3pt}{0pt}{red},
    %colbacktitle={block-gray},
    coltitle={black},
    fonttitle={\bfseries},
    attach title to upper={\\},
    lines before break=3
    }{S}
    
\newenvironment{Bsp}[2]{\vspace{0.2cm}\begin{BspI}{#1}{#2}}{\end{BspI}\vspace{0.2cm}}
\newenvironment{Satz}[2]{\vspace{0.2cm}\begin{SatzI}{#1}{#2}}{\end{SatzI}\vspace{0.2cm}}
\newenvironment{Def}[2]{\vspace{0.2cm}\begin{DefI}{#1}{#2}}{\end{DefI}\vspace{0.2cm}}


\setlist[description]{font=\normalfont\itshape\space}

\usepackage{listings}
\usepackage[scaled=0.7]{beramono}


\renewcommand\thesection{\arabic{chapter}.\arabic{section}}
\renewcommand\thesubsection{\arabic{chapter}.\arabic{section}.\arabic{subsection}}
\renewcommand\thesubsubsection{\arabic{chapter}.\arabic{section}.\arabic{subsection}.\arabic{subsubsection}}

% Outer Join Symbols
\def\ojoin{\setbox0=\hbox{$\bowtie$}%
  \rule[-.02ex]{.25em}{.4pt}\llap{\rule[\ht0]{.25em}{.4pt}}}
\def\leftouterjoin{\mathbin{\ojoin\mkern-5.8mu\bowtie}}
\def\rightouterjoin{\mathbin{\bowtie\mkern-5.8mu\ojoin}}
\def\fullouterjoin{\mathbin{\ojoin\mkern-5.8mu\bowtie\mkern-5.8mu\ojoin}}
\newenvironment{select}[1]{\sigma_{#1} \left(}{\right)}
\newenvironment{project}[1]{\pi_{#1} \left(}{\right)}
\newenvironment{aggregate}[2]{\Gamma_{#1~\#~#2} \left(}{\right)}
\newcommand{\rel}[1]{\MakeUppercase{\text{#1}}}
\newcommand{\join}[1]{\underset{\substack{#1}}{\bowtie}}
\newcommand{\semijoin}[1]{\underset{\substack{#1}}{\ltimes}}
\newcommand{\naturaljoin}{\bowtie}
\newcommand{\antisemijoin}[1]{\underset{\substack{#1}}{~\overline{\ltimes}~}}


\sloppy

%% Pseudocode - Imports
\usepackage{algorithm}
\usepackage{algpseudocode}
\floatname{function}{algorithm}
\algnewcommand\algorithmicforeach{\textbf{for each}}
\algdef{S}[FOR]{ForEach}[1]{\algorithmicforeach\ #1\ \algorithmicdo}
\newenvironment{breakablealgorithm}
  {% \begin{breakablealgorithm}
   \begin{center}
     \refstepcounter{algorithm}% New algorithm
     \hrule height.8pt depth0pt \kern2pt% \@fs@pre for \@fs@ruled
     \renewcommand{\caption}[2][\relax]{% Make a new \caption
       {\raggedright\textbf{\fname@algorithm~\thealgorithm} ##2\par}%
       \ifx\relax##1\relax % #1 is \relax
         \addcontentsline{loa}{algorithm}{\protect\numberline{\thealgorithm}##2}%
       \else % #1 is not \relax
         \addcontentsline{loa}{algorithm}{\protect\numberline{\thealgorithm}##1}%
       \fi
       \kern2pt\hrule\kern2pt
     }
  }{% \end{breakablealgorithm}
     \kern2pt\hrule\relax% \@fs@post for \@fs@ruled
   \end{center}
  }